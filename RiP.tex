\documentclass[a4paper,11pt]{article}

%%%%%%%%%%%%%%%%%%%%%%%%%%%%%%%%%%%
%       FRONT MATTER
%%%%%%%%%%%%%%%%%%%%%%%%%%%%%%%%%%%

\title{Computing Bianchi $p$-adic $L$-functions and applications in Iwasawa theory}
\author{Marc Masdeu, Bharathwaj Palvannan and Chris Williams}
\date{}
\pagestyle{headings}
\usepackage{fourier}

\usepackage{enumitem}
%\usepackage{ulem}
\usepackage{amsmath}

%\let\mathcal\undefined
%\DeclareMathAlphabet{\mathcal}{OMS}{cmsy}{m}{n}

\let\OLDthebibliography\thebibliography
\renewcommand\thebibliography[1]{
\OLDthebibliography{#1}
\setlength{\parskip}{0pt}
\setlength{\itemsep}{0pt plus 0.3ex}
}


% Line spacing: 1 is normal
%\renewcommand{\baselinestretch}{1.1}


%Preamble
\usepackage[lmargin=1.2in, rmargin=1.2in, tmargin=1.3in, bmargin=1.3in]{geometry}
\usepackage{preamble}
%\ProvidesPackage{preamble-ams}
%\usepackage{preamble-ams}
\usepackage{xypic}

\usepackage{etoolbox}
\newtoggle{notes}
\toggletrue{notes}
%\togglefalse{notes} 
%\usepackage{comment}
\usepackage{verbatim}
\iftoggle{notes}{
%% INCLUDE NOTES

\newcommand{\CWnote}[1]{
 \begin{color}{red}
 \marginpar{$\spadesuit$} #1
  --- Chris
 \end{color}
}

\newcommand{\BP}[1]{
	\begin{color}{blue}
		\marginpar{$\heartsuit$} #1
		--- Bharath
	\end{color}
}


\newcommand{\thoughts}[1]{
#1
}
}{


\newcommand{\CWnote}[1]{

}
\newcommand{\thoughts}[1]{
}
}


\usepackage{fancyhdr}

%Heading Title
\lhead{\emph{Explicit Bianchi $p$-adic $L$-functions}}
\rhead{\emph{Masdeu, Palvannan \& Williams}}


\newcommand{\lb}{\\[8pt]}

%%%%%%%%%%%%%%%%%%%%%%%%%%%%%%%%%%%%%%%%%%%%%%%%%%%%
%
%                                                                MACROS
% 
%                          (To make sure we agree the same notation everywhere and can change it easily)
%
%%%%%%%%%%%%%%%%%%%%%%%%%%%%%%%%%%%%%%%%%%%%%%%%%%%%


\newcommand{\fl}{\mathfrak{l}}
\newcommand{\cO}{\mathcal{O}}
\newcommand{\q}{\mathfrak{q}}
\newcommand{\pibar}{\overline{\pi}}

\newcommand{\unitsize}{\#\roi_K^\times}
\newcommand{\roihat}{\widehat{\roi}}
\newcommand{\FQ}{_{K/\Q}}



\newcommand{\Tbc}{T_{\mathrm{bc}}}


%Rigid spaces
\newcommand{\W}{\mathcal{W}}
\newcommand{\Wfull}{\mathscr{W}}

\newcommand{\WW}{\mathcal{W}}
\newcommand{\E}{\mathcal{E}}
\newcommand{\CC}{\mathcal{C}}

\newcommand{\cyc}{{\mathrm{cyc}}}

\newcommand{\Dz}{\mathcal{D}^0}
\newcommand{\Dl}{\mathscr{D}^0}
\newcommand{\Dla}{\mathcal{D}}
\newcommand{\DD}{\mathscr{D}}
\newcommand{\VV}{\mathscr{V}}

\newcommand{\rigidA}{\mathcal{A}^0}
\newcommand{\laA}{\mathcal{A}}

%Cohomology
\newcommand{\hc}{\h^1_{\mathrm{c}}}
\newcommand{\ssh}{^{\leq h}}
\newcommand{\Y}{Y_1(\n)}

%Localisations
\newcommand{\locx}{_{\m_x}}
\newcommand{\locl}{_{\m_\lambda}}


\DeclareMathOperator{\Tor}{Tor}
\DeclareMathOperator{\Ext}{Ext}
\DeclareMathOperator{\Sp}{Sp}
\DeclareMathOperator{\Gal}{Gal}

\newcommand{\mel}{\mathrm{Mel}}

\newcommand{\Epar}{\E_{\mathrm{par}}}
\newcommand{\Ref}{\mathrm{ref}}
\newcommand{\bc}{\mathrm{bc}}

  \DeclareFontFamily{U}{wncy}{}
    \DeclareFontShape{U}{wncy}{m}{n}{<->wncyr10}{}
    \DeclareSymbolFont{mcy}{U}{wncy}{m}{n}
\DeclareMathSymbol{\Sha}{\mathord}{mcy}{"58}

\numberwithin{equation}{section}
\newcommand{\fn}{\mathfrak{n}}



\newcommand{\Galois}[2]{\Gal({#1}/{#2})}
\newcommand{\Par}{\mathrm{par}}




\titleformat{\subsubsection}[runin]% runin puts it in the same paragraph
{\normalfont\itshape}% formatting commands to apply to the whole heading
{\thesubsubsection.}% the label and number
{0.5em}% space between label/number and subsection title
{}% formatting commands applied just to subsection title
[.\hspace*{4pt}]% punctuation or other commands following subsection title




\begin{document}
%%%%%%%%%%%%%%%%%%%%%%%%%%%%%%%%%%%%%%%%%%%%%%%%%
%
%                   TITLE AND ABSTRACT
%
%%%%%%%%%%%%%%%%%%%%%%%%%%%%%%%%%%%%%%%%%%%%%%%%%

\maketitle


\begin{abstract}
\begin{comment}
Let $\f$ be a cuspidal eigenform of weight 2 over an imaginary quadratic field $K$. In this paper, we use the overconvergent cohomology of Bianchi groups to explicitly compute $p$-adic $L$-functions attached to $\f$. We give several applications to Iwasawa theory: in particular, we provide new examples where Coates and Sujatha's psuedo-nullity conjecture holds for rational elliptic curves.  --- including the first examples in rank 1.
\end{comment}
Using overconvergent cohomology of Bianchi groups, we present an algorithm that explicitly computes the $2$-variable $p$-adic $L$-function associated to a weight two cuspidal eigenform with small slope over an imaginary quadratic field with class number one. The algorithm builds on earlier work of the third author establishing a constructive proof of Steven's control theorem for overconvergent lifts of Bianchi modular forms. Our implementation of this algorithm crucially relies on adaptating efficient code built by the first author (along with Guitart and Sengun) that computes the required overconvergent lifts\BP{what is the right verb here? Adapt? Also is the attribution to Guitart and Sengun accurate }. Implementing this algorithm allows us to compute numerical examples of interest in the topic of higher codimension Iwasawa theory. In these numerical examples, the validity of the appropriate Iwasawa main conjectures implies that the Pontraygin dual of the fine Selmer group is pseudo-null.
\end{abstract}


\section{Introduction}


We first introduce the notations required to describe our results. Let $p$ be an odd prime number. Let $K$ be an imaginary quadratic field\footnote{so $K$ equals $\Q(\sqrt{-D})$, where $D$ belongs to $\{1,2,3,7,11,19,43,67,163\}$.} with class number one, where the prime $p$ splits as $(p) = \pri \bar{\pri}$. Let $K_\pri$, $K_{\overline{\pri}}$ denote the unique $\Z_p$-extensions of $K$ unramified outside $\pri$, $\overline{\pri}$ respectively. Let $G_\pri$, $G_{\overline{\pri}}$ denote $\Galois{K_\pri}{K}$, $\Galois{K_{\overline{\pri}}}{K}$ respectively. Let us choose topological generators $g_\pri$, $g_{\overline{\pri}}$  of $G_\pri$, $G_{\overline{\pri}}$ respectively. Let $\widetilde{K}_\infty$ denote the compositum of all the $\Z_p$-extensions of $K$. Let $\widetilde{G}_\infty$ denote $\Galois{\widetilde{K}_\infty}{K}$. Recall that we have the following isomorphisms of topological groups:
\begin{align*}
\widetilde{G}_\infty \cong \Z_p^2, \qquad G_{\pri} \cong \Z_p, \qquad G_{\overline{\pri}} \cong \Z_p.
\end{align*}


Since $p$ does not divide the class number of $K$, the natural restriction maps $\widetilde{G}_\infty \twoheadrightarrow G_\pri$ and $\widetilde{G}_\infty \twoheadrightarrow G_{\overline{\pri}}$ allow us to identify $\widetilde{G}_\infty$ with  $G_\pri \times G_{\overline{\pri}}$. This further allows to consider the following isomorphism of topological $\Z_p$-algebras:
\begin{align} \label{eq:iwid}
\Z_p[[\widetilde{G}_\infty]] &\cong \Z_p[[x,y]], \\
g_\pri &\rightarrow x+1, \notag \\
g_{\overline{\pri}} &\rightarrow y+1 \notag
\end{align}


Throughout this paper, we shall fix the isomorphism in (\ref{eq:iwid})  and denote the $2$-variable Iwasawa algebra appearing in (\ref{eq:iwid}) by $\Lambda$.

Let $\psi: \Gal\left({\overline{K}}/{K}\right) \rightarrow \Z_p^\times$ be a finite order continuous character. \BP{I will think of the best way to say this? So that we can incorporate Teichmuller twists, but also so that we can incorporate characters of conductor prime to p}


Let $O_K$ denote the ring of integers of $K$. Let $\n$ be an ideal in $O_K$ satisfying the following condition: \BP{do we need $p$ to exactly divide the level}.
\begin{enumerate}[style=sameline, style=sameline, align=left,label=(\textsc{$p$-stab}) --- , ref=(\textsc{$p$-stab})]
\item\label{hyp:pstab} $\n \subset pO_K$.
\end{enumerate}
Let $\Gamma$ denote the congruence subgroup $\Gamma_0(\n)$ of $\SLt(\mathcal{O}_K)$. Let $f$ be a weight $2$ \BP{Is there a prefix before weight? Is it parallel weight $2$ or just weight $2$?} Bianchi cuspidal eigenform of level $\Gamma$ with Hecke eigenvalues $\{a_I \in \Z_p: I\subset \roi_K\}$, satisfying the following condition:
\begin{enumerate}[style=sameline, style=sameline, align=left,label=(\textsc{Small slope}) --- , ref=(\textsc{Small slope})]
\item\label{hyp:ss} The $p$-adic valuations of the eigenvalues $a_\pri$, $a_{\overline{\pri}}$ for the actions of the $U_\pri$, $U_{\overline{\pri}}$ operators on $f$ are strictly less than $1$.
\end{enumerate}

Let $\D_2$ denote the space of $\Z_p$-valued $p$-adic distributions on $\Z_p^2$, endowed with a \textit{weight two} action of $\Gamma$. Let $\Symb_\Gamma(\D_2)$ denote the corresponding space of overconvergent modular symbols valued in $\D_2$. There exists an overconvergent modular symbol $\mu_{f,\psi}$ in $\D_2$ associated to the tuple $(p,f,\psi)$. See Section \ref{sec:l-functions} for a description of the interpolation properties satisfied by $\mu_{f,\psi}$ as predicted by Coates and Perrin-Riou. Recall also that the Mahler transform (\BP{or is it called the Amice transform}) provides a natural isomorphism $\D_2 \cong \Lambda$ of topological $\Z_p$-algebras.\\

The purpose of our paper is to present an algorithm that computes the Mahler transform (denoted $L_{p,f,\psi}(x,y)$ here) of $\mu_{f,\psi}$ evaluated at the path $\left\{0 \rightarrow \infty \right\}$. To wit, we have
\begin{align}\label{eq:p-adic-identify}
\D_2 &\cong \Lambda, \notag \\
\mu_{f,\psi}\left(\left\{0 \rightarrow \infty \right\}\right) &\leftrightarrow L_{p,f,\psi}(x,y).
\end{align}

The elements appearing in the identification given in equation (\ref{eq:p-adic-identify}) are referred to as the $p$-adic $L$-function. The application to Iwasawa theory stems from the fact that it is $L_{p,f,\psi}(x,y)$ that appears in the formulation of the Iwasawa-Greenberg main conjecture,  and is conjectured to generate the characteristic divisor in $\Lambda$ of the Pontryagin dual of a Selmer group. The construction of the Selmer group and the formulation of the main conjecture, both due to Greenberg, are recalled in Section \ref{}. \BP{describe later}

\subsection{Description of the algorithm}

The principal objects appearing in our algorithm are the following:

\begin{itemize}
\item [---] The overconvergent modular symbols valued in $\D_2$: $\Symb_\Gamma(\D_2)$
\item[---] The parabolic cohomology group valued in distributions: $H^1_{\Par}(\Gamma,\D_2)$
\item[---] The parabolic cohomology group with trivial coefficients $H^1_{\Par}(\Gamma,\Z_p)$.
\end{itemize}

Each of these objects are naturally endowed with an action of the  Hecke algebra (for weight $2$ and level $\Gamma$) and are related to each other by Hecke-equivariant maps as illustrated below:
\begin{align}
\xymatrix{
\Symb_{\Gamma}(\D_2) \ar[r]^{\delta}&  H^1_{\Par}\left(\Gamma,\D_2 \right) \ar[d]^{\varrho} \\
& H^1_{\Par}\left(\Gamma,\Z_p \right).
} & \qquad \qquad
\xymatrix{
\mu_{f,\psi} \ar[r]^{\delta}&  [\Phi_f] \ar[d]^{\varrho} \\
& [\phi_f].
}
\end{align}

These maps are described in Section \ref{}. We will simply remark here that the map $\delta$ is a \textit{connecting homomorphism} in group theory, whereas the map $\varrho$ is obtained from the natural \textit{specialization map} $\D_2 \twoheadrightarrow \Z_p$.

We now outline the main steps involved in our algorithm.\BP{will work on this}

\begin{enumerate}[style=sameline, style=sameline, align=left,label=(\textsc{Step 1}) --- , ref=(\textsc{Step 1})]
\item\label{step1} \underline{\emph{Constructing a cocycle $[\varphi_f]$ with trivial coefficients}}.
\end{enumerate}
In \ref{step1}, we construct a cocycle $\varphi$ in $Z^1(\Gamma,\Z_p)$ such that its corresponding cohomology class in $H^1_{\Par}\left(\Gamma,\Z_p \right)$ equals $[\varphi_f]$.


\begin{enumerate}[style=sameline, style=sameline, align=left,label=(\textsc{Step 2}) --- , ref=(\textsc{Step 2})]
\item\label{step2} \underline{\emph{Constructing an overconvergent lift $[\Phi_\f]$}}
\end{enumerate}

Since $f$ satisfies \ref{hyp:ss},  a control theorem \cite[Thm.~6.10]{Wil17} shows that there exists a unique cohomology class $[\Phi_f]$ in $H^1_{\Par}\left(\Gamma,\D_2 \right)$, stable under the action of the Hecke algebra, such that $\rho([\Phi_f])$ equals $[\phi_f]$.

In \ref{step2}, we construct a cocycle $\Phi$ in $Z^1(\Gamma,\D_2)$ such that its corresponding cohomology class in $H^1_{\Par}\left(\Gamma,\D_2 \right)$ equals $[\Phi_f]$.


\begin{enumerate}[style=sameline, style=sameline, align=left,label=(\textsc{Step 3}) --- , ref=(\textsc{Step 3})]
\item\label{step3} \underline{\emph{Inverting the connecting homomorphism $\delta$ to obtain $\mu_{f,\psi}$}}
\end{enumerate}

\begin{enumerate}[style=sameline, style=sameline, align=left,label=(\textsc{Step 4}) --- , ref=(\textsc{Step 4})]
\item\label{step4} \underline{\emph{Computing the $2$-variable power series $L_p(x,y)$ in $\Lambda$}}.
\end{enumerate}

\subsection{Explicit Bianchi $p$-adic $L$-functions}
To compute the $p$-adic $L$-functions of \cite{Loe14}, there is an algorithm of the third author. The main result of \cite{Wil17} was a construction of a $p$-adic $L$-function attached to very general classes of Bianchi modular forms -- that is, modular forms over $K$ -- using a generalisation of Stevens' theory of \emph{overconvergent modular symbols}.  A particular feature of this approach, as explored for classical modular forms in \cite{PS11}, is its amenability to computation.

To $E/\Q$, one can attach a classical modular form $f$ under modularity, and through base-change this corresponds to a Bianchi modular form $\f$. By assumption, this will have level $N\cO_F$ prime to $p$, but for each prime $\pri|p$, there are two $\pri$-stabilisations to level $\pri N\cO_F$, and hence four $p$-stabilisations to level $pN\cO_F$. The four $p$-adic $L$-functions attached to these stabilisations in \cite{Wil17} are precisely those of \cite{Loe14}. We give a sketch of the explicit construction in \S\ref{sec:bianchi p-adic l-functions}.

We actually use a modified version of the construction of \cite{Wil17}. Existing code is much more developed for computations with arithmetic group cohomology, rather than modular symbols, so we develop a cohomological version of the construction. This approach, however, introduces new theoretical complications, which we explain and treat in the main text. In particular, it requires explicitly inverting the natural map from modular symbols to group cohomology.

[Include remarks on Rob Pollack and co, and in particular emphasise the difficulty of computing 2-variable things: they restrict to index in the 100s, it takes them many hours and they compute to precision 7. We should do better?]


\subsection{Concluding Remarks}

\begin{enumerate} 
\item Explain why  we restrict to $K$ to have class number one.
\item Explain here why we choose to work with group cohomology instead of modular symbols.
\item Explain here all the places where the choices are being made (I think only initially in choosing generators) and emphasize all the objects that are canonical and don't depend on any choices (like the measure etc).
\item Explain here how the precision is related to storing moments
\item Explain how we verify our code by existing theoretical results, relating it to 1-variable p-adic L-functions of Pollack. And perhaps, mu-invariant results of Hida.
\item Analyse the run time of our algorithm. What are our practical limitations in computing things.
\item Explain how our code compares to existing code of Pollack - Harron and co.
\item Write down all the attributions here --- Stevens, Pollack, Greenberg etc. 
\end{enumerate}

\subsection*{Acknowledgements}
This research was supported through the programme "Research in Pairs" by the Mathematisches Forschungsinstitut Oberwolfach in 2018; the MFO also supported this project by enabling M.M.\ and C.W.\ to visit B.P.\ whilst he was there in 2019 as a Leibniz fellow. We are grateful to the MFO for twice providing a stimulating place to do research. We would like to thank Aurel Page for helpful conversations about his code for computing presentations of arithmetic groups. The authors were supported by the following grants: [INSERT HERE] (M.M.); [INSERT HERE] (B.P.); a Heilbronn Research Fellowship and EPSRC Postdoctoral Fellowship [EP/T001615/1] (C.W.).


%%%%%%%%%%%%%%%%%%%%%%%%%%%%%%%%%%%%%%%%%%%%%%%%%%%%%%%%%%%%%%%%%%%%
%%%%%%%%%%%%%%%%%%%%%%%%%%%%%%%%%%%%%%%%%%%%%%%%%%%%%%%%%%%%%%%%%%%%
\section{Bianchi modular forms and \texorpdfstring{$p$}{p}-adic \texorpdfstring{$L$}{L}-functions}
\label{sec:bianchi p-adic l-functions}
We recap the underlying theory of Bianchi modular forms, which are a natural generalisation of classical modular forms -- that is, automorphic forms for $\GLt/\Q$ -- to $\GLt/K$, where $K$ is an imaginary quadratic field. Throughout this paper, we take $K$ to have class number 1 for simplicity (in line with our later computations). Let $\mathcal{O}_K$ be the ring of integers. 

\subsection{Background}
For each congruence subgroup $\Gamma \subset \SLt(\mathcal{O}_K),$ there is a finite-dimensional $\C$-vector space $S_{2}^K(\Gamma)$ of `weight 2 Bianchi cusp forms', defined as harmonic vector-valued functions
\[
    \f = (\f_0,\f_1,\f_2) : \uhs \longrightarrow \C^3
\]
 on the upper half-space $\uhs := \C\times\R_{>0}$ satisfying a suitable transformation property under an action of $\Gamma.$ See \cite[\S1]{Wil17} for a more detailed exposition of all of this (standard) theory in our conventions (noting that $S_2^K(\Gamma)$ for us is denoted $S_{0,0}(\Gamma)$ \emph{op.\ cit}.). 
 
 There is a Hecke action on $S_2^K(\Gamma)$, given by double coset operators and indexed by ideals of $\mathcal{O}_K$. 

\subsubsection{$L$-functions and $p$-adic $L$-functions}\label{sec:l-functions}
If $\f \in S_2^K(\Gamma)$ is an eigenform with Hecke eigenvalues $\{a_I : I\subset \roi_K\}$, and $\chi$ is a Hecke character of $K$, one defines the $L$-function 
\[
	L(\f,s) = \sum_{I \subset \roi_K}a_I \chi(I)N(I)^{-s}
\] 
as the corresponding Dirichlet series (see \cite[\S1.2]{Wil17}). This sum converges absolutely for $\mathrm{Re}(s)>>0$ and admits analytic continuation to arbitrary $s\in\C$. Morever, there exists a complex period $\Omega_{\f} \in \C^\times$ such that
\begin{equation}\label{eqn:algebraic}
	\frac{L(\f,\chi,1)}{(2\pi i)^2\Omega_{\f}} \in \overline{\Q}
\end{equation}
for all finite order Hecke characters $\chi$ (see \cite[\S8]{Hid94}). The conjecture of Coates--Perrin-Riou then predicts the existence of a \emph{$p$-adic $L$-function} interpolating these algebraic numbers. More precisesly, this should be a $p$-adic distribution $\mu_{\f}$ on 
\[
	\mathrm{Gal}(K^{\mathrm{ab},p}/K) \cong \cl_K(p^\infty) \defeq K^\times\A_K^\times/\C^\times \widehat{\roi}_K^{\times,(p)} \cong (\roi_K\otimes_{\Z}\Zp)^\times/\roi_K^\times,
\]
where $K^{\mathrm{ab},p}$ is the maximal abelian extension of $K$ unramified outside $p$, the first isomorphism is given by class field theory, and the second follows from our assumption of class number 1. The required interpolation property is that for all finite order Hecke characters of $p$-power conductor -- which are naturally characters of $\cl_K(p^\infty)$ -- we have
\begin{equation}\label{eqn:interpolation}
	\mu_f(\chi) \defeq \int_{\cl_K(p^\infty)}\chi(z)d\mu_{\f}(z) = (*)\frac{L(\f,\chi,1)}{(2\pi i)^2\Omega_{f}},
\end{equation}
where $(*)$ is a precise interpolation factor (described explicitly in this case in \cite[Thm.~3.12]{BW17} with $r=0$). Moreover, $\mu_{\f}$ should satisfy a precise growth property. 


\subsubsection{Base-change}

Of particular importance to us is the existence of a \emph{base-change} map. Let $E/\Q$ be an elliptic curve of conductor $N$ corresponding to a newform $f \in S_2(\Gamma_0(N))$ under modularity. We can consider $E$ instead to have coefficients over $K$. As predicted by Langlands, there exists a Bianchi newform $f/K$, the \emph{base-change of $f$ to $K$}, whose $L$-function is equal to $L(E/K,s)$. We have $f/K \in S_2^K(\Gamma_0(\fn))$, where $\Gamma_0(\fn) \subset \SLt(\mathcal{O}_K)$ is the subgroup of matrices that are upper-triangular mod $\fn$, and where $\fn|N\mathcal{O}_K$ is the conductor of $E/K$. (Note that if $N$ is coprime to the discriminant of $K/\Q$, we have $\fn = N\mathcal{O}_K$). The Hecke eigenvalues of $f/K$ can be described simply in terms of those of $f$ (see e.g.\ \cite[\S7.2]{BW18}). Note that
\[
    L(f/K,s) = L(E/K,s) = L(E/\Q,s)L(E/\Q,\chi_K,s) =  L(f,s)L(f,\chi_K,s),
\]
where $\chi_K$ is the quadratic Hecke character whose kernel cuts out $K/\Q$.




%==================================
\subsection{Bianchi modular symbols}
We are interested in computational aspects of Bianchi modular forms, and particularly their $L$-functions, but the analytic definition is impractical for such purposes. Fortunately, the space $S_2^K(\Gamma)$ admits a computational avatar via the space of \emph{modular symbols}. Let 
\[
    \Delta_0 \defeq \mathrm{Div}^0(\Proj(K))
\]
denote the space of `paths between cusps' in $\uhs$, and let $V$ be any right $\SLt(K)$-module. Fix an ideal $\fn$ and let $\Gamma := \Gamma_0(\fn) \subset\SLt(K)$. We define the space of \emph{$V$-valued modular symbols for $\Gamma$} to be the space
\[
    \symb_{\Gamma}(V) \defeq \mathrm{Hom}_\Gamma(\Delta_0,V)
\]
of functions satisfying the $\Gamma$-invariance property that
\[(\phi|\gamma)(\delta)\defeq \phi(\gamma \delta)|\gamma = \phi(\delta)\hspace{12pt} \forall \delta\in\Delta_0, \gamma\in\Gamma,\]
where $\Gamma$ acts on the cusps by $\smallmatrd{a}{b}{c}{d}\cdot r = (ar+b)/(cr+d).$  Note that this space is entirely algebraic in nature, and far more amenable to computations.

Now let $\f \in S_2^K(\Gamma)$. To $\f$ one may attach an explicit differential form $\delta_{\f}$ on $\uhs$ as follows: let $(z,t)$ be a co-ordinate on $\uhs$, and note that $dz, d\overline{z}, dt$ span the $\C$-valued 1-forms on $\uhs$. Then define
\[
    \delta_{\f} := \f_0(z,t)dz - \f_1(z,t)dt - \f_2(z,t)d\overline{z} \in \Omega^1(\uhs)
\]
(see e.g.\ \cite{CrWh94}). For $r,s \in \Proj(K),$ the map
\begin{align*}
    \phi_{\f} : \Delta_0 &\longrightarrow \C,\\
    \{r\to s\} &\longmapsto \int_{r}^s \delta_f,
\end{align*}
is well-defined and the transformation property satisfied by $\f$ ensures it is $\Gamma$-invariant, thus giving an element $\phi_f \in \Symb_\Gamma(\C).$ 

The space $\Symb_\Gamma(\C)$ admits an action of the Hecke operators, indexed by ideals of $\cO_F$ and generated by the operators $T_{\fl}$ for $\fl\nmid\fn$ prime and $U_{\fl}$ for $\fl|\fn$. Of particular importance are the $U_{\pri}$-operators for $\pri|p$, defined as
\begin{align*}
	\phi|U_{\pri} \{r\to s\} &= \sum_{a=0}^{p-1} \phi\big|\smallmatrd{1}{a}{0}{\pi}\{r-s\} = \sum_{a=0}^{p-1}\phi\left\{\tfrac{r+a}{\pi} \to \tfrac{s+a}{\pi}\right\},
\end{align*}
where $\pri = (\pi)$. More generally, if the action of $\Gamma$ on $V$ extends to an action of the semigroup
\[
\Sigma_0(p) \defeq \{\smallmatrd{a}{b}{c}{d} \in M_2(\roi_K\otimes_{\Z}\Zp) :  ad-bc \neq 0, p\nmid a, p|c\},
\] 
then the formula $\phi|U_{\pri} = \sum_{a = 0}^{p-1}\phi|\smallmatrd{1}{a}{0}{\pi}$ defines an action of $U_{\pri}$ on $\Symb_\Gamma(V)$.

\begin{proposition}
 The resulting map 
    \[
        \iota : S_2^K(\Gamma) \longrightarrow \Symb_\Gamma(\C)
    \]
    is injective and induces a splitting
    \[
        \symb_\Gamma(\C) \cong S_2^K(\Gamma) \oplus \mathrm{Eis}_2^K(\Gamma).
    \]
    For a prime $\fl\nmid \fn$ of norm $\ell$, the Hecke operator $T_{\fl}$ acts on $\mathrm{Eis}_2^K(\Gamma)$ as multiplication by $\ell+1$.
\end{proposition}
\begin{proof}
    See, for example, \cite{Wil17} for the injection. For part (ii), observe that by \cite[Lemma 8.2]{BW17} we have a Hecke-equivariant isomorphism
    \[
        \symb_\Gamma(\C) \cong \mathrm{H}^1_{\mathrm{c}}(Y_\Gamma, \C),
    \]
    where $Y_\Gamma := \Gamma\backslash\uhs$, and the compactly supported cohomology admits a well-understood splitting $\mathrm{H}^1_{\mathrm{cusp}}(Y_\Gamma,\C) \oplus \mathrm{Eis}_\Gamma(\C)$ (see \cite[\S3.2.5]{Har87}). Moreover, under the injection $\iota$, the space $S_2^K(\Gamma)$ is mapped isomorphically onto $\mathrm{H}^1_{\mathrm{cusp}}(Y_\Gamma,\C)$. The other direct summand corresponds to Bianchi Eisenstein series, giving the claimed action of Hecke operators.
\end{proof}

The passage from an eigenform $\f$ to $\phi_{\f}$ thus encodes much of the interesting algebraic data attacehed to $\f$. In particular, it retains the Hecke action, the Hecke eigenvalues and thus also sees the $L$-function of $\f$. In particular, in \cite{CrWh94} an explicit formula is given relating $\phi_{\f}\{0\to\infty\}$ to the critical value $L(\f,1)$. A twisted version, computing $L(\f,\chi,1)$ for finite order Hecke characters of $K$, is contained in \cite[Prop.\ 2.8]{Wil17}.




%%%%%%%%%%%%%%%%%%%%%%%%%%%%%%%%%%%%%%%%%%%%%%%%%
%     Ovcvgt modular symbols
%%%%%%%%%%%%%%%%%%%%%%%%%%%%%%%%%%%%%%%%%%%%%%%%%
\subsection{Overconvergent modular symbols}


We briefly recap the main ideas in the construction of \cite{Wil17}. The idea behind the overconvergent modular symbol construction  is that to $p$-adically interpolate the (algebraic parts of the) values $L(\f,\chi,1)$, we can use the connection to modular symbols, and $p$-adically interpolate these instead. To do this, we require algebraic coefficients. By \cite[\S8]{Hid94}, there exists a finite extension $L/\Qp$ such that we may consider $\phi_{\f}/\Omega_{\f} \in \symb_{\Gamma}(L)$, where $\Omega_f$ is the same complex period from \eqref{eqn:algebraic}, which can then be interpolated $p$-adically via \emph{overconvergent modular symbols}.

For such an interpolation to exist, it is crucial to pass to a level subgroup $\Gamma \subset \Gamma_0(p\cO_F).$ If $(p)|\fn$, then $\Gamma_0(\fn) \subset \Gamma_0(p\cO_F)$ already; if not, then it is necessary to `$p$-stabilise' to ensure this condition holds (see for example \cite[\S2.4]{BW18}). We now assume $\Gamma \subset \Gamma_0(p\cO_F)$ without further comment.

We pass to an (infinite-dimensional) coefficient module; namely, let $L/\Qp$ be a finite extension, let $\A_2(L)$ denote the space of convergent power series on $\Zp^2$, and let
\[
    \D_2(L) := \mathrm{Hom}_{\mathrm{cts}}(\A_2(L),L),
\]
the space of \emph{$p$-adic analytic distributions on $\Zp^2$.} As $\Gamma \subset \Gamma_0(p),$ the left action of $\Gamma$ on $\A_2(L)$ by
\[
    \smallmatrd{a}{b}{c}{d} \cdot g(x,y) = g\left(\tfrac{b+dx}{a+cx}, \tfrac{\overline{b} + \overline{d}y}{\overline{a}+\overline{c}y}\right),
\]
is well-defined and induces dually a right action on $\D_2(L).$ The space of \emph{overconvergent modular symbols} is $\symb_{\Gamma}(\D_2(L)).$

Dualising the inclusion of $L$ into $\A_2(L)$ gives a surjection $\D_2(L) \rightarrow L$ of $\Gamma$-modules, and hence a (Hecke-equivariant) map
\[
    \rho : \symb_\Gamma(\D_2) \rightarrow \symb_\Gamma(L).
\]
Whilst this map will have a huge kernel, crucially, the Hecke action allows us to control it.

\begin{theorem}\emph{\cite[Thm.~6.10]{Wil17}} \label{thm:control theorem}
Let $\f \in S_2^K(\Gamma)$ be an eigenform, with $U_{\pri}\f = \alpha_{\pri}\f$ for each $\pri|p$. If $v_{\pri}(\alpha_{\pri}) < 1$ for all $\pri$, then the restriction of $\rho$ to the $\f$-eigenspaces of the Hecke operators is an isomorphism.
\end{theorem}

If $\f$ satisfies this condition, we say it has \emph{small slope}. If $\f$ is such a form, then the theorem says that there is a \emph{unique} $\Phi_{\f} \in \symb_\Gamma(\D_2(L))$ lifting $\phi_{\f}/\Omega_{\f}.$ 


%%%%%%%%%%%%%%%%%%%%%%%%%%%%%%%%%%%%%%%%%%%%%%%%%
%     p-adic L-function
%%%%%%%%%%%%%%%%%%%%%%%%%%%%%%%%%%%%%%%%%%%%%%%%%
\subsection{The $p$-adic $L$-function of a Bianchi modular form}\label{padic lfunction section}

Let $\f$ be a small slope Bianchi eigenform with associated overconvergent modular symbol $\Phi_{\f}$. Recall from \S\ref{sec:l-functions} that the $p$-adic $L$-function should be a $p$-adic distribution on $\cl_K(p^\infty) = (\Zp^\times)^2/\roi_K^\times$. Note that there is a natural inclusion $\A(\cl_K(p^\infty),L) \subset \A_2(L)$
of the subspace of analytic functions on $\cl_K(p^\infty)$, and thus a natural restriction map $\D_2(L) \to \D(\cl_K(p^\infty))$.

\begin{definition} 
Define the \emph{$p$-adic $L$-function of $\f$} to be
\[
	\mu_{\f} \defeq \Phi_{\f}\{0 \to \infty\}\big|_{\cl_K(p^\infty)}.
\]
\end{definition}

By \cite[Thm.~7.4]{Wil17}, this is actually in the much smaller subspace of \emph{locally analytic distributions}, and can thus be evaluated at finite order Hecke characters. It then satisfies the expected interpolation property \eqref{eqn:interpolation} and has the expected growth condition. In the small slope case, this defines $\mu_{\f}$ uniquely.

%%%%%%%%%%%%%%%%%%%%%%%%%%%%%%%%%%%%%%%%%%%%%%%%%%%%%%%%%%%%%%%%%%%%
%%%%%%%%%%%%%%%%%%%%%%%%%%%%%%%%%%%%%%%%%%%%%%%%%%%%%%%%%%%%%%%%%%%%
\section{Computing \texorpdfstring{$p$}{p}-adic \texorpdfstring{$L$}{L}-series from modular symbols}
In this section, we show how to compute the $p$-adic $L$-series of a Bianchi modular form from its $p$-adic $L$-function (as a distribution on $\cl_K(p^\infty)$). Since we are working in the case where $p$ is split, this essentially just the product of two copies of the theory for classical modular forms, as found in \cite[\S9]{PS11}, and which we first briefly recall.

%%%%%%%%%%%%%%%%%%%%%%%%%%%%%%%%%%%%%%%%%%%%%%%%
\subsection{The classical case}
%%%%%%%%%%%%%%%%%%%%
\subsubsection{Distributions on $\Zp^\times$}

For simplicity, we assume that $p \geq 3$, though the case $p = 2$ can be obtained with very little (and completely standard) modification. In the classical case, the $p$-adic $L$-function is naturally a distribution on the Galois group $\mathrm{Gal}_p \defeq \mathrm{Gal}(\Q^{\mathrm{ab},p\infty}/\Q)$, which by class field theory is isomorphic to the narrow ray class group 
\[
	\cl_{\Q}^+(p^\infty) = \Q^\times\backslash \A^\times/\R_{>0}\widehat{\Z}^{\times,(p)} \cong \Zp^\times.
\]
Let now $f \in S_2(\Gamma)$ be a classical eigenform (for $\GLt/\Q$) with $U_pf = \alpha_p f$ with $v_p(\alpha_p) < 1$. By \cite{PS11}, there is a unique overconvergent modular symbol $\Phi_f \in \Hom_\Gamma(\Div^0(\Proj(\Q)),\D)$ attached to $f$, where $\D$ is the space of distributions on $\Zp$. Then $\Phi_f\{0 \to \infty\} \in \D$ is by definition a distribution on $\Zp$, and we obtain a distribution $\mu_f$ on $\Zp^\times$ by restriction. This distribution is, by \cite{PS11}, the $p$-adic $L$-function attached to $f$, and is entirely encoded in its moments 
\[
\left\{\mu_f(z^j) \defeq \int_{\Zp^\times} z^j d\mu_f(z) : j \geq 0\right\}.
\] We have a simple description of the moments of $\mu_f$ in terms of $\Phi_f$, and thus, in particular, we can compute them by knowing $\Phi_f$.

\begin{proposition}\label{prop:apply Up}
	If $\Phi_f$ is the overconvergent modular symbol attached to a small slope classical cuspidal eigenform, then
	\[
		\Phi\{0\rightarrow\infty\}(z^j \mathbf{1}_{a+p\Zp}(z)) = \alpha_p^{-1}\Phi\left\{\tfrac{a}{p} \rightarrow \infty\right\}\big((a+pz)^j\big),
	\]
     where $\alpha_p$ is the $U_p$-eigenvalue of $f$. In particular, we recover
     \[
     	\mu_f(z^j) = \alpha_p^{-1}\sum_{a=1}^{p-1} \Phi\left\{\tfrac{a}{p} \rightarrow \infty\right\}\big((a+pz)^j\big).
     \]
\end{proposition}
\begin{proof}
Apply $\alpha_p^{-1}U_p$, which acts as 1 on $\Phi_f$, and note that if $a = b$, then
\[
\Phi\{b/p \rightarrow \infty\}((b+pz)^j\mathbf{1}_{a+p\Zp}(b+pz)) = 0. \qedhere
\]
The final statement follows since $\mathbf{1}_{\Zp^\times} = \sum_{a=1}^{p-1}\mathbf{1}_{a+p\Zp}$.
\end{proof}

\subsubsection{Passing to $p$-adic $L$-series}
The distribution $\mu_f$ is completely canonical. Even in higher weight situations, it is canonical up to a fixed choice of periods. For practical purposes, however, to compute $p$-adic $L$-series from $\mu_f$ we wish to break this into $p-1$ pieces via the decomposition 
\begin{equation}\label{eqn:Zp decomp}
	\Zp^\times = (\Z/p\Z)^\times \times (1+p\Zp),
\end{equation}
 and then (non-canonically) identify each one with a distribution on $\Zp$ (depending on a choice of topological generator for $1+p\Zp$). On the algebraic side, this corresponds to considering the cyclotomic $\Zp$-extension of $\Q$, identifying its Galois group $\mathcal{G}$ with $\Zp$ via the choice of a topological generator, and studying Selmer groups over $\mathcal{G}$.


\begin{notation}
	In \eqref{eqn:Zp decomp}, denote projection to the first and second factors by $z \mapsto \{z\}$ and $z \mapsto \langle z\rangle$ respectively. 
\end{notation}

We can identify distributions $\mu$ on $\Zp^\times$ with analytic functions on \emph{weight space} 
\[
	\W = \mathrm{Hom}_{\mathrm{cts}}(\Zp^\times, \C_p^\times),
\]
 which -- as a rigid space -- is $p-1$ copies of the open unit disc. Given an element $\phi$ in weight space, we can write $\phi$ as $\{\phi\}\langle\phi\rangle$, where $\{\phi\}$ is a homomorphism $(\Z/p\Z)^\times \rightarrow \Cp^\times,$ and the disc in $\W$ in which $\phi$ lives is completely determined by $\{\phi\}$. From $\mu_f$ we then obtain $p-1$ analytic functions on the open unit disc, one for each character of $(\Z/p\Z)^\times$. Each has a power series representation, which is what we compute.

Fix now a character $\psi$ on $(\Z/p\Z)^\times$, corresponding to some fixed disc in $\W$. Let $T$ be a parameter on this open unit disc. The part of the (analytic) function attached to $\mu_f$ defined over this disc is defined as
\[
	L_p(\mu, \psi, T) \defeq \int_{\Zp^\times} \psi(z)(T+1)^{\log_{\gamma}(\langle z\rangle)} d\mu_f(z),
\]
 where $\gamma=p+1$, our fixed choice of topological generator for $1 + p\Zp$. (Of course, for $z \in 1 + p\Zp$, we have $\log_\gamma(z) = \log_p(z) / \log_p(\gamma)$). Writing $z = \{z\}\langle z\rangle$ and expanding the log, we obtain the power series representation $L_p(\mu,\psi,T) = \sum_{n\geq 0} d_n(\psi)T^n$, where $d_n(\psi)$ is defined as
\[
  d_n(\psi) =\sum_{a=0}^{p-1}\psi(a)  \int_{a+p\Z_p} \left[\sum_{j\geq 0} c_j^{(n)}\left(\frac{z}{\{a\}}-1\right)^j \right]d\mu(z).
\]
Here, $c_j^{(n)}$ is defined by from the equation
\begin{equation}\label{cjn}
  \binom{\log_{\gamma}(z+1)}{n} = \sum_{j\geq 0} c_j^{(n)} z^j,
\end{equation}
Note also that for $a$ prime to $p$, we have $\int_{a+p\Zp}h(z) d\mu(z) = \int_{a+p\Zp}h(z)d\Phi\{0\to\infty\}(z)$, that is, restriction to $\Zp^\times$ is already built in. For further details on all of the above, see \cite[\S9]{PS11}.

%%%%%%%%%%%%%%%%%%%%%%%%%%%%%%%%%%%%%%%%%%%%%%%%%%%%%%%%%%%%%%%%%%%%
\subsection{The Bianchi case}
Now we turn to the Bianchi case. Let $f$ be a small slope classical cuspidal Bianchi eigenform, and $\Phi_f \in \Symb_{\Gamma}(\D_2)$ the attached overconvergent modular symbol. As outlined in \S\ref{sec:l-functions}, the $p$-adic $L$-function of $f$ is most naturally a distribution $\mu$ on $\cl_K(p^\infty) \cong (\roi_K\otimes_{\Z}\Zp)^\times/\roi_K^\times,$ where in the last assertion we are assuming class number one for simplicity. Since $p$ is split, we identify this with $(\Zp^\times \times \Zp^\times)/\roi_K^\times$. Evaluating $\Phi_f$ at $\{0\rightarrow\infty\}$, we obtain a distribution on all of $\Zp^2$. Inside $\A_2$ is the subspace of functions with support in $\Zp^\times \times \Zp^\times$ that are invariant under $\roi_K^\times$, and we pass to the $p$-adic $L$-function $\mu_f$ by restricting $\Phi_f\{0\to\infty\}$ to this subspace.

One can consider the direct analogue of the weight space above, that is, considering continuous homomorphisms $\Zp^\times \times \Zp^\times \rightarrow \Cp^\times$, and show that this decomposes as the disjoint union of $(p-1)^2$ products of open unit discs, parametrised by characters of $(\roi_K/p)^\times = (\Z/p)^\times \times (\Z/p)^\times$. By directly generalising the approach above, given a measure $\mu$ on $\cl_K(p^\infty)$ and such a character $\psi$, one can define an associated (two-variable) analytic function $L_p(\mu,\psi,T_1,T_2)$ on the corresponding product of open discs. Because the $p$-adic $L$-function actually lives on $\cl_K(p^\infty)$ rather than $(\roi_K\otimes_{\Z}\Zp)^\times$, such a function can only be defined when $\psi$ descends to the quotient $(\roi_K/p)^\times/\roi_K^\times$.

Fix such a character $\psi$. Writing $p\roi_K = \pri\pribar$, we see that $\psi = \psi_{\pri}\psi_{\pribar}$, where $\psi_{\pri}$ is the restriction to $(\roi_K/\pri)^\times$. We also have parameters $z_{\pri}, z_{\pribar}$ on $\cl_K(p^\infty)$, and $T_{\pri}, T_{\pribar}$ on the product of open discs. Using the same arguments as above, we find the following.

\begin{proposition}
The $p$-adic $L$-series attached to $\psi$ and $\mu$ is
\[
	L_p(\mu,\psi,T_{\pri},T_{\pribar}) = \sum_{m\geq 0}\sum_{n\geq 0} d_{m,n}(\psi) T_{\pri}^mT_{\pribar}^n,
\]
where -- for $c_i^{(m)}$ as in \eqref{cjn} -- we define
\begin{align*} 
    d_{m,n}(\psi) = \sum_{a=1}
^{p-1}\sum_{b=1}^{p-1} &\psi_{\pri}\big(a\big) \psi_{\pribar}\big(b\big) \\
\times &\int_{(a+p\Zp) \times (b+p\Zp)} \bigg[ \sum_{i\geq 1} \sum_{j\geq 1} c_i^{(m)} c_j^{(n)} \left(\frac{z_{\pri}}{\{a\}}-1\right)^i \left(\frac{z_{\pribar}}{\{b\}}-1\right)^j \bigg]d\mu(\mathbf{z}) .
\end{align*}
\end{proposition}

%%%%%%%%%%%%%%%%%%%%%%%%%%%%%%%%%%%%%%%%%%%%%%%%%%%%%%%%%%%%%%%%%%%%
\subsubsection{Obtaining power series from the moments of \texorpdfstring{$\mu$}{mu}}
As before, it is simple to obtain the moments of $\mu_f$ by applying the $U_p$ operator. Write $V_{a,b}$ for the open compact set $(a+p\Zp)\times(b+p\Zp) \subset (\roi_K\otimes_{\Z}\Zp)^\times$. We compute that
\begin{align*}
	\int_{V_{a,b}}f(z_{\pri},z_{\pribar}) d\mu(\mathbf{z}) &\defeq \Phi\{0\rightarrow\infty\}(f(z_{\pri},z_{\pribar})\mathbbm{1}_{V_{a,b}}) \\
    &= \alpha_{p}^{-1}\Phi\{c/p \rightarrow \infty\}\big(f(c + pz_{\pri}, \overline{c} + pz_{\pribar})\big),
    \end{align*}
where $c = c_{a,b} \in \roi_K$ is such that
\[
	c \equiv a \pmod{\pri},\quad \overline{c}\equiv b\pmod{\pribar}.
\]
To see this, one applies the operator $\alpha_p^{-1}U_{p\roi_K} = \alpha_p^{-1}U_{\pri}U_{\pribar}$, which acts as the identity on $\Phi$, and note that the indicator function kills all but the $a$ term of $U_{\pri}$ and the $b$ term of $U_{\pribar}$, corresponding to the $c$ term of $U_p$; see \cite[\S7.1]{Wil17} for more details. Since for us $f$ is a polynomial function, it is simple to compute this value by taking a linear combination of the moments of $\Phi\{c/p\rightarrow \infty\}$.

%%%%%%%%%%%%%%%%%%%%%%%%%%%%%%%%%%%%%%%%%%%%%%%%%%%%%%%%%%%%%%%%%%%%
\subsubsection{Twists}
Twisting by characters of $p$-power conductor is built into the definitions above, but we can also twist by finite order characters of prime-to-$p$ conductor. Consider a finite order character $\chi$ of conductor $(\mathfrak{d}) = \mathfrak{D} \subset \roi_K$ prime to $p$. From $\Phi$, one defines a twisted symbol
\[
	\Phi_\chi \defeq \sum_{b \newmod{\mathfrak{D}}} \chi(b) \bigg[\Phi\{b/\mathfrak{d} \rightarrow \infty\}\bigg| \matrd{1}{b}{0}{\mathfrak{d}}\bigg],
\]
then computes $L_p(\mu,\psi\chi,T) \defeq L_p(\mu_\chi,\psi,T),$ where $\mu_\chi \defeq \Phi_\chi\{0\rightarrow \infty\}|_{\cl_K(p^\infty)}$. This case is treated in \cite[\S3.4]{BW17}.










%%%%%%%%%%%%%%%%%%%%%%%%%%%%%%%%%%%%%%%%%%%%%%%%%%%%%%%%%%%%%%%%%%%%
%%%%%%%%%%%%%%%%%%%%%%%%%%%%%%%%%%%%%%%%%%%%%%%%%%%%%%%%%%%%%%%%%%%%

%%%%%%%%%%%%%%%%%%%%%%%%%%%%%%%%%%%%%%%%%%%%%%%%%%%%%%%%%%%%%%%%%%%%
%%%%%%%%%%%%%%%%%%%%%%%%%%%%%%%%%%%%%%%%%%%%%%%%%%%%%%%%%%%%%%%%%%%%

\section{Rephrasing via arithmetic cohomology}
The above gives a complete algorithm for constructing $p$-adic $L$-series from Bianchi modular symbols. For practical reasons, however, this space would be hard to compute in, since this requires presenting $\mathrm{Div}^0(\Proj(K))$ as a $\Z[\Gamma]$-module (and, in particular, solving the word problem in such a presentation). Instead, we work with the arithmetic cohomology groups $\h^1(\Gamma,\D_2)$, for which extensive implementation already exists. This approach might also generalise more naturally to different settings, for example, pursuing the constructions of \cite{Bar15} over \emph{real} quadratic fields, where modular symbols themselves do not exist and one is forced to work with higher degree cohomology groups.

\subsection{Definition and basic properties}
See Marc's papers (which one explains it best, Marc?), or  \cite{PP09}; include things such as $U_p$ operators.



%\subsection{Computing \texorpdfstring{$p$}{p}-adic \texorpdfstring{$L$}{L}-series from cohomology}

A downside of computing with arithmetic cohomology over modular symbols is that we are not free to evaluate at the same range of divisors. In particular, we have a map
\begin{align*}
	\delta : \symb_{\Gamma}(\D_2) &\longrightarrow \h^1(\Gamma,\D_2)\\
    \Phi &\longmapsto \big(\varphi: \gamma \mapsto \Phi\{\gamma\cdot\infty \rightarrow \infty\}\big).
\end{align*}

 Given $\varphi$, we can thus read off the values $\Phi\{r \to s\}$ \emph{only} for $r,s$ equivalent to the cusp $\infty$. This poses a problem for our algorithm, since we need to evaluate at pairs $\{a/p \to \infty\}$, where $a$ is coprime to $p$, and in general $a/p$ will \emph{not} give the same cusp as $\infty$. To obtain the information we need from $\varphi$, then, requires a careful study of the map $\delta$.

From the general theory, the kernel and cokernel of $\delta$ are Eisenstein; in particular, it is an isomorphism on the cuspidal part. It follows that there is a \emph{unique} cuspidal lift $\Phi$ of $\varphi$ under $\delta$. In the next section, we show how to explicitly invert $\delta$ to obtain this class $\Phi$ from $\varphi$, and thus how to obtain the $p$-adic $L$-function from $\varphi$.


%%===============================================================
%%		MOTIVATION

\subsection{Explicit inversion of $\delta$}

We now give an explicit and computable recipe for inverting $\delta$ on cuspidal arithmetic cohomology classes, taking care to be precise at every step.

\subsubsection{Motivation: a snake diagram}
The map $\delta$ can be realised in cohomology as 
\[
\delta : \h^0(\Gamma,\mathrm{Hom}(\Delta_0,\D_2) \longrightarrow \h^1(\Gamma,\D_2),
\]
and is the connecting map in a long exact sequence given by the snake lemma. In particular, for a right $\Gamma$-module $\mathcal{M}$, let:
\begin{itemize}
	\item $C^i(\mathcal{M}) \defeq C^i(\Gamma,M)$ = $\mathcal{M}$-valued $i$-cochains for $\Gamma$,
	\item $Z^i(\mathcal{M}) \defeq$ $\mathcal{M}$-valued $i$-cocycles for $\Gamma$,
	\item and $B^i(\mathcal{M}) \defeq$ $\mathcal{M}$-valued $i$-coboundaries for $\Gamma$. 
\end{itemize}
Explicitly, as we are using right modules, a 1-cocycle is a map $z : \Gamma \to \mathcal{M}$ such that $z(\gamma_1\gamma_2) = z(\gamma_1) + z(\gamma_2)|\gamma_1^{-1}$. Also write $(C^i/B^i)(\mathcal{M})$ for the $i$-cochains modulo the $i$-coboundaries.

Recall that $\Delta_0 = \mathrm{Div}^0(\Proj(K))$, and let $\Delta = \mathrm{Div}(\Proj(K)) = \Z[\Proj(K)]$. Now, the degree map gives a short exact sequence $0 \to \Delta_0 \to \Delta \to \Z \to 0$, and hence -- for any right $\Gamma$-module $M$ -- a short exact sequence
\[
	0 \to M \longrightarrow \Hom(\Delta,M) \longrightarrow \Hom(\Delta_0,M) \to 0,
\]
of $\Gamma$-modules, identifying $M \cong \Hom(\Z,M')$, and where $\Gamma$ act on $\Hom(\Delta,M')$ by $(\phi|\gamma)(D) = \phi(\gamma D)|\gamma$. This gives rise to a diagram
\[
	\xymatrix@C=10mm@R=4mm{
		& & & \h^0(\Gamma,\Hom(\Delta_0,M))\ar[d] &\\
			 & \frac{C^0}{B^0}\big(\Gamma,M\big)\ar[r]\ar[d]   & \frac{C^0}{B^0}\big(\Gamma,\Hom(\Delta,M)\big)\ar[r]\ar[d] &  \frac{C^0}{B^0}\big(\Gamma,\Hom(\Delta_0,M)\big)\ar[r]\ar[d] &0\\
	0 \ar[r]  & Z^1(\Gamma,M)  \ar[r]\ar[d] & Z^1(\Gamma,\Hom(\Delta,M))  \ar[r] & Z^1(\Gamma,\Hom(\Delta_0,M)) &\\
	& \h^1(\Gamma,M) & & & \\
}
\]
from which we obtain the snake exact sequence
\[
\cdots \to \h^0(\Gamma,\Hom(\Delta,M)) \xrightarrow{\alpha} \h^0(\gamma,\Hom(\Delta_0,M)) \xrightarrow{\delta} \h^1(\Gamma,M) \xrightarrow{\beta} \h^1(\Gamma,\Hom(\Delta,M)) \to \cdots.
\]
In proving that this is exact, one takes an element in $\ker(\beta)$ -- such as our class $\varphi$ -- and explicitly realises it in the image of $\delta$. Thus we can lift $\delta$ by pursuing this diagram chase.


\subsubsection{An explicit formula for inverting $\delta$}
In practice, we will work at the level of cocycles, and this will depend on choices made throughout the algorithm (whilst the corresponding cohomology classes, and the $p$-adic $L$-functions, will not). Suppose we are given a cohomology class $\varphi \in H^1(M)$ represented by a cocycle $\varphi_0$. If $M = \D_2$ and $\varphi$ arises as the lift of a cuspidal classical eigenclass, then it is in $\ker(\beta)$. Then:

\begin{proposition}\label{prop:stab coboundaries}
If $\varphi \in \ker(\beta)$, then for any $c_i \in \Proj(K)$, the restriction of $\varphi_0$ to $\mathrm{Stab}_\Gamma(c_i)$ is a coboundary in $Z^1(\mathrm{Stab}_\Gamma(c_i),M)$; explictly, there exists $v_i \in M$ such that
\[
	\varphi_0(\gamma) = v_i|\gamma^{-1} - v_i
\]
for all $\gamma \in \mathrm{Stab}_\Gamma(c_i)$.
\end{proposition}
\begin{proof}
	This is essentially a consequence of Shapiro's lemma, but with slightly more explicit control on the coboundaries. The map $\beta$ is induced by the map of cocycles that sends the cocycle $\varphi_0 : \Gamma \to M$ to
	\[
		\varphi_0 \longmapsto \bigg(\beta\left(\varphi_0\right) : \gamma \mapsto \left[r \mapsto \varphi_0(\gamma)\right]\bigg),
	\]
	where $r$ is any element of $\Proj(K)$ (recalling that $\Delta = \Z[\Proj(K)]$ is freely generated such $r$). In particular, each $\gamma \in \Gamma$ is just sent to a constant function in $\Hom(\Delta,M)$. For $\varphi \in \ker(\beta)$, we must have $\beta(\varphi_0)$ is a coboundary, and hence that there exists $v \in \Hom(\Delta,M)$ such that 
	\[
		\beta(\varphi_0)(\gamma) = v|\gamma^{-1} - v.
	\]
	In particular, we have 
	\[
		\beta(\varphi_0)(\gamma)(r) = v(\gamma^{-1} r)|\gamma^{-1} - v(r) = \varphi_0(\gamma)
	\]
	for all $c \in \Proj(K)$, by definition of $\beta(\varphi_0)$. For $c_i$ as above and $\gamma \in \mathrm{Stab}_\Gamma(c_i)$, we have
	\[
		\varphi_0(\gamma) = v(c_i)|\gamma^{-1} - v(c_i) = v_i|\gamma^{-1} - v_i,
	\]
	for $v_i \defeq v(c_i) \in M$, as required.
	%
	%For the converse, suppose $\varphi_0$ is in $\ker(\oplus \mathrm{res}_{c_i})$; then we have $v_i$ as above for each cusp representative $c_i$. But this is enough to reconstruct $v \in \Hom(\Delta,M)$ such that $\beta(\varphi_0)(\gamma) = v|\gamma^{-1} - v$ for every $\gamma \in \Gamma$, since every cusp is $\Gamma$-equivalent to one of the $c_i$. In particular, we can define
	%\[
	%	v(r) = v(\gamma_r c_{i(r)}) = 
	%\]
	%\[
	%	\beta(\varphi_0)(\gamma)(r) = \varphi_0(\gamma) = v(r)|\gamma^{-1} - v(r) = v(g_r c_{i(r)})|\gamma^{-1} - v(g_r c_{i(r)})
	%\]
\end{proof}

Let now $c_1, ..., c_t \in \Proj(K)$ be a complete set of representatives for the cusps, and for each $r \in \Proj(K)$, let $g_r \in \Gamma$ and $i(r) \in \{1,...,t\}$ be such that
\[
	g_r \cdot r = c_{i(r)}. 
\]
Assuming $\varphi \in \ker(\beta)$, let $v_1, ..., v_t \in \D_2$ be the distributions arising from Proposition \ref{prop:stab coboundaries}. We can compute the $v_i$ by computing the action of $g$ as a linear operator on $\D_2$ (up to some precision), and then solving the resulting linear system for a sufficiently large set of elements of the stabiliser\footnote{In practice, this is actually more subtle than it appears: in the linear system, the variables will appear with varying degrees of precision, due to the filtration appearing in the explicit lifting theorem.}.
	
Again motivated by the snake lemma, we now define $\widetilde{\Phi} \in \Hom_\Gamma(\Delta,\D_2) = \h^0(\Gamma,\Hom(\Delta,\D_2))$ by setting

	\begin{align*}
		\widetilde{\Phi} : \Proj(K) &\longrightarrow \D_2,\\
				r = g_r^{-1}c_{i(r)} &\longmapsto \varphi(g_r)\big|g_r + v_{i(r)}\big|g_r
	\end{align*}
	and extending linearly, and $\Phi = \alpha(\widetilde{\Phi}) : \Delta_0 \longrightarrow \D_2$ defined by
	\[
		\Phi\{r\to s\} \defeq \widetilde{\Phi}(s) - \widetilde{\Phi}(r).
	\]


%%===============================================================
%%		ALGORITHM

%\subsubsection{Algorithm for inverting $\delta$}
%Summarising the above, we arrive at the following recipe for inverting $\delta$. Let $\varphi \in \h^1(\Gamma,\D_2)$ be a lift of a cuspidal classical eigenclass. In particular, it will be parabolic, and for each cusp $r \in \Proj(K)$, it will vanish under the restriction maps $\mathrm{Res}_r : \h^1(\Gamma,\D_2) \to \h^1(\mathrm{Stab}_{\Gamma}(r),\D_2)$.
%\begin{enumerate}
%	\item Find representatives $c_1 = \infty, c_2, ..., c_t \in \Proj(K)$ of the cusps $\Gamma\backslash\Proj(K)$ of $\Gamma$.
%	\item For each $r \in \Proj(K)$, let $c_{i(r)}$ be the equivalent representative, and let $g_r \in \Gamma$ be such that in $\Proj(K)$, we have
%	\[
%			g_r \cdot r = c_{i(r)}.
%	\]
%	\item Explicitly represent $\mathrm{Res}_{c_i}(\varphi)$ as a coboundary in $\h^1(\mathrm{Stab}_{\Gamma}(c_i), \D_2)$. In particular, find $v_1, ..., v_t \in \D_2$ such that 
%	\[
%		\varphi(g) = v_i|g^{-1} - v_i
%	\]
	
	
	\begin{proposition}
		The map $\Phi$ gives a well-defined element of $\Symb_{\Gamma}(\D_2)$ with $\delta(\Phi) = \varphi$.
	\end{proposition}
	\begin{proof}
		The map $\Phi$ is linear in $r \to s$, since it is defined as the difference $\widetilde{\Phi}(s) - \widetilde{\Phi}(r)$. It is mapped to $\varphi$ under $\delta$; setting $r = \gamma\cdot\infty$ and $s = \infty$, we have $g_r = \gamma^{-1}$, $g_s = 1$, and $c_{i(r)} = c_{i(s)} = c_1$. Then by definition,
		\begin{align*}
			\delta(\Phi)(\gamma) \defeq \Phi\{\gamma\cdot\infty \to \infty\} &= \varphi(1)|1 + v_1 - \varphi(\gamma^{-1})|\gamma^{-1} - v_1|\gamma^{-1}\\
									&= \varphi(\gamma) + \left[v_1 - v_1|\gamma^{-1}\right],
		\end{align*}
		using that $\varphi(1) = 0 = \varphi(\gamma\gamma^{-1}) = \varphi(\gamma) + \varphi(\gamma^{-1})|\gamma^{-1}$. The term in the square brackets is a coboundary; thus the cocycle $\delta(\Phi)$ represents the same cohomology class as $\varphi$.
		
		It remains to show that $\Phi$ is $\Gamma$-equivariant. Let $\gamma \in \Gamma$. Note that if $g_r \cdot r = c_{i(r)}$, then $g_r\gamma^{-1} \cdot \gamma r = c_{i(r)}$, so that $c_{i(\gamma r)} = c_{i(r)}$ and $g_{\gamma r} = g_r\gamma^{-1}$. Then
		\begin{align*}
			\widetilde{\Phi}(\gamma r) &= \varphi(g_{\gamma r})|g_{\gamma r} + v_{i(\gamma r)}|g_{\gamma r}\\
					 &= \varphi(g_r \gamma^{-1})|g_r\gamma^{-1} + v_{i(r)}|g_r\gamma^{-1}\\
						&= \varphi(g_r)|g_r\gamma^{-1} + \varphi(\gamma^{-1})|g_r^{-1}g_r\gamma^{-1} + v_{i(r)}|g_r\gamma^{-1}\\
						&= \widetilde{\Phi}(r)|\gamma^{-1} + \varphi(\gamma^{-1})|\gamma^{-1}.
		\end{align*}
		The second term is independent of $r$, so cancels in the difference $\widetilde{\Phi}(s) - \widetilde{\Phi}(r)$. It follows that
		\[
			\Phi\{\gamma r \to \gamma s\}|\gamma = \widetilde{\Phi}(s)|\gamma - \widetilde{\Phi}(r)|\gamma = \Phi\{r\to s\},
		\]
		as required.
	\end{proof}

In general, the map $\Phi$ thus defined is not an eigensymbol. In particular, whilst there is a unique cuspidal lift $\Phi_{\mathrm{cusp}}$ of $\varphi$ under $\delta$, the map $\Phi$ can be any element of $\Phi_{\mathrm{cusp}} + \mathrm{ker}(\delta)$. However, we have:

\begin{proposition}
	For all sufficiently large primes $\mathfrak{l}$ of $K$, coprime to $p\fn$, the symbol
	\[
		\Phi_{\f} \defeq \frac{T_{\mathfrak{l}} - \ell - 1}{a_{\mathfrak{l}} - \ell + 1} \Phi \in \Symb_\Gamma(\D_2)
	\] 
	is the uniquely determined overconvergent (cuspidal) eigensymbol mapped to $\varphi$ under $\delta$, where $\ell$ is the norm of $\mathfrak{l}$ and $a_{\mathfrak{l}}$ is the $T_{\mathfrak{l}}$-eigenvalue of the Bianchi modular form $\f$.
\end{proposition}
\begin{proof}
By the long exact sequence given by the snake lemma, the kernel of $\delta$ is given by the image of $\mathrm{Hom}_\Gamma(\Delta,\D_2)$ in $\Symb_\Gamma(\D_2)$, or, more precisely, the Eisenstein subspace. We know that for prime $\mathfrak{l}\nmid p\fn$ of norm $\ell$, the Hecke operator $T_{\mathfrak{l}}$ acts on the Eisenstein subspace by $\ell + 1$ (see, for example, \cite[Rem.~5.2]{PS11} for this in the rational case; more generally, it can be obtained by studying $\mathrm{Hom}(\Delta,\D_2)$ as a Hecke-module). For sufficiently large $\mathfrak{l}$, the Hasse bound implies that $a_{\mathfrak{l}} \neq \ell + 1$. By the remarks above, the operator $T_{\mathfrak{l}} - \ell -1$ kills any Eisenstein contribution, and thus acts as a projector onto the cuspidal subspace. Now renormalising by $a_{\mathfrak{l}} - \ell - 1$ gives a cuspidal eigensymbol in $\delta^{-1}(\varphi)$. Such a symbol is unique by strong multiplicity one combined with Theorem \ref{thm:control theorem}.
\end{proof}
%Henceforth we shall simply renormalise and write $\Phi$ in place of $\Phi_{\mathrm{cusp}}$.


\begin{comment}
\subsection{old}



Let thus $\varphi$ be a cocycle representing a cohomology class $[\varphi]\in H^1(\Gamma,\D_2)$, and assume that its restriction to each cusp $c$ is trivial. Choose representatives $\infty=c_1,\ldots c_t\in \mathbb{P}^1(\Q)$ for the cusps of $\Gamma$, and set, for $\gamma_i,\gamma_j\in \Gamma$ and representatives $c_i$ and $c_j$,
\[
\Phi\{\gamma_j c_j\rightarrow \gamma_i c_i\} = \gamma_j\varphi(\gamma_j^{-1})-\gamma_i\varphi(\gamma_i^{-1}).
\]
\begin{proposition}
With the previous definition, we have:
\begin{enumerate}
    \item $\Phi$ is well defined.
    \item $\Phi\in \Symb_{\Gamma}(\D_2)$,
    \item $\delta(\Phi)=\varphi$.
\end{enumerate}
\end{proposition}
\begin{proof}
Suppose that we replace $\gamma_i$ with $\gamma_i'$ and $\gamma_j$ with $\gamma_j'$, in a way that
\[
\gamma_ic_i=\gamma_i'c_i,\quad \gamma_jc_j=\gamma_j'c_j.
\]
Then the difference between the corresponding $\varphi$ evaluations is
\[
\left(\gamma_j'\varphi((\gamma_j')^{-1})-\gamma_j\varphi(\gamma_j^{-1})\right) - \left(\gamma_i'\varphi((\gamma_i')^{-1})-\gamma_i\varphi(\gamma_i^{-1})\right),
\]
so it suffices to prove that
\[
\gamma_i\varphi(\gamma_i^{-1}) = \gamma_i'\varphi((\gamma_i')^{-1}).
\]
But remember that $\gamma_ic_i =\gamma_i'c_i$, which means that $\gamma_i'= \gamma_i g$, for some $g\in\operatorname{Stab}_{\Gamma}(c_i)$. In this case, we have
\[
\gamma_i g \varphi(g^{-1} \gamma_i^{-1}) - \gamma_i \varphi(\gamma_i^{-1}) = \gamma_i g \varphi(g^{-1}).
\]
We will show that $\varphi(g)=0$ for all $g\in\operatorname{Stab}_{\Gamma}(c_i)$. Since $\varphi$ is parabolic, we know there is a map $f\colon \mathbb{P}^1(\Q_p)\to \D_2$ such that $\varphi(g) = gf(g^{-1}P) - f(P)$ for all $P\in \mathbb{P}^1(\Q_p)$, and without loss of generality we may assume that $f(c_i)=0$. Therefore,
\[
\varphi(g)=gf(c_i)-f(c_i) = 0 -0=0.
\]

To prove the second claim, the definition clearly makes $\Phi$ to be linear on $\Div^0(\mathbb{P}^1(\Q))$, so it is enough to check $\Gamma$-equivariance. This follows from the following calculation:
\begin{align*}
\Phi\{g\gamma_jc_j\rightarrow g\gamma_ic_i\}&=\varphi(g\gamma_j)-\varphi(g\gamma_i) = 
g\gamma_j\varphi(\gamma_j^{-1}g^{-1}) - g\gamma_i\varphi(\gamma_i^{-1}g^{-1})\\
&=\left(g\gamma_j\varphi(\gamma_j^{-1}) + g\varphi(g^{-1})\right) -\left(g\gamma_i\varphi(\gamma_i^{-1}) + g\varphi(g^{-1})\right)\\
&= g\left(\gamma_j\varphi(\gamma_j^{-1}) - \gamma_i\varphi(\gamma_i^{-1})\right)\\
&=g\Phi\{\gamma_jc_j\rightarrow \gamma_ic_i\}.
\end{align*}

The third claim is obvious: since $\infty=c_1$, we have
\[
(\delta\Phi)(\gamma)=\Phi\{\infty\rightarrow\gamma\infty\} = \varphi(1)-\gamma\varphi(\gamma^{-1})-\varphi(1)=\varphi(\gamma).
\]
\end{proof}

The modular symbol thus defined is not an eigensymbol, in general. Since $\delta$ is Hecke-equivariant, we have:
\[
\delta(T(f) - f) = 0,
\]
and so $T(f) = f + \epsilon$, where $\epsilon$ is an Eistenstein symbol. In order to obtain an eigensymbol it suffices to apply for example $T_\ell-\ell-1$ for any prime $\ell$.
      
% The problem, however, is that it is difficult to make the inverse of $\delta$ explicit. In particular, given an overconvergent arithmetic class $\varphi$, from the definition we only know how to evaluate $\delta^{-1}(\varphi)$ at divisors of the form $\gamma\cdot\infty \rightarrow \infty$, where $\gamma \in \Gamma$. The cusps $0$ and $\infty$, however, are not $\Gamma$-equivalent, so it is far from clear how to extract the data $\delta^{-1}(\varphi)\{0\rightarrow\infty\}$ from $\varphi$. For clarity, we first explain how to do this for $\GLt/\Q$, which is similar but simpler.
\end{comment}


\subsubsection{Remark: an alternative formula using $U_p$}
In certain cases, there are even more explicit formula -- inspired by Prop.~\ref{prop:apply Up} -- for computing $p$-adic $L$-functions from the knowledge only of $\varphi$ (without inverting $\delta$ to compute the whole modular symbol $\Phi$). We highlight this only in the rational case for simplicity, though the same ideas go through for Bianchi modular forms too.


Let $\varphi \in \h^1(\Gamma,\D_2)$ be an overconvergent eigenclass. Recall if $\delta(\Phi) = \varphi$, then $\varphi(\gamma) = \Phi\{\gamma\infty \to \infty\}$, and in particular we can directly evaluate $\Phi$ at any pair of cusps that are both equivalent to $\infty$. The idea is then to use the Hecke operators to rewrite our target values -- namely, the moments $\Phi\{a/p \to \infty\}((a+pz)^j)$ -- in terms of values of $\Phi\{r\to \infty\}$, where each $r$ is $\Gamma$-equivalent to $\infty$, and which we can then read off from $\varphi$. For $\Gamma = \Gamma_0(N)$, this is true if and only if $r = a/N$, where $a$ is prime to $N$. We must thus express the $\Phi\{a/p \to \infty\}$ in terms of values $\Phi\{\beta/N \to \infty\} = \varphi(\gamma),$ where $\gamma = \smallmatrd{\beta}{*}{N}{*} \in \Gamma_0(N)$.


Suppose for simplicity that $N = q \neq p$ is prime. Then we have:
\begin{proposition}
Let $d_q$ denote the order of $q$ in $(\Z/p\Z)^\times$, and write $a_q$ for the $U_q$-eigenvalue of $\Phi$. Then we have
\begin{align*}
\big(1 -a_q^{-d_q}&q^{jd_q}\big)\Phi\left\{\tfrac{a}{p}\rightarrow\infty\right\}((a+pz)^j) =
\\
& a_p^{-1} \sum_{m = 0}^{d_q-1} a_q^{-m}q^{jm}%\lambda_1^{-m_1}q_1^{m_1j}\cdots \lambda_r^{-m_r}q_r^{m_rj} 
\sum_{\substack{\beta \in (\Z/Np\Z)^\times\\ \beta \equiv a/q^m\newmod{p}}} \Phi\left\{\tfrac{\beta}{qp}\rightarrow \infty\right\}\big((\beta + qpz)^j\big).
\end{align*}
\end{proposition}
We will not give the full proof, but instead observe that one obtains this by applying the operator $\alpha_q^{-1}U_q$, which acts as the identity, to force $q$ to be in the denominator. At each iteration, there is one problem term where the $q$ in the denominator cancels, and we end up with only $p$ (rather than $pq$) in the denominator. Iterating this process at the problem term, after $d_q$ iterations one arrives at the original expression $\Phi\{a/p \to \infty\}((a+pz)^j)$ multiplied by $a_q^{-d_q}q^{jd_q}$, giving the claimed formula.

We can now find an element $\gamma = \smallmatrd{\beta}{*}{pq}{*} \in \Gamma_0(N)$, and $\gamma\infty = \beta/pq$; thus $\Phi\{\beta/pq - \infty\} = \varphi(\gamma)$ and we can compute this directly, without inverting $\delta$. Similar formula hold for $N$ a product of primes, and also in the Bianchi setting, where now we consider moments attached to a pair of integers $(j,k)$.

\begin{remark}
Note that if $j \neq 0$ (or, in the Bianchi setting, when either $j \neq 0$ or $k \neq 0$) the scalar $1-a_q^{-d_q}q^{jd_q}$ is always non-zero, and thus we can always recover the \emph{higher} moments required to define the $p$-adic $L$-series without inverting $\delta$. However, when $j = 0$ (or $j=k=0$), it is possible for the scalar to vanish. In particular, if $f$ is attached to an elliptic curve $E$ of rank 1 with good reduction at $p$, then the sign of the functional equation forces the existence of some prime $q$ such that $a_q(E) = 1$, and then the scalar vanishes. This leads to the curious phenomenon where we can compute all but the first moment; and yet the first moment is precisely the value of the \emph{classical} modular symbol we started with. Thus via this method, in this case we can compute all of the overconvergent moments, but not the classical starting moment!
\end{remark}



%%%%%%%%%%%%%%%%%%%%%%%%%%%%%%%%%%%%%%%%%%%%%%%%%%%%%%%%%%%%%%%%%%%%
%%%%%%%%%%%%%%%%%%%%%%%%%%%%%%%%%%%%%%%%%%%%%%%%%%%%%%%%%%%%%%%%%%%%
\section{The algorithms}

\subsection{Computing with the classical cohomology}
There are well-established routines for computing classical cohomology classes attached to elliptic curves over number fields, as used by the first author and his collaborators in in [CITE MARC'S PAPERS]. These rely on computing a presentation of the group $\Gamma$, using existing Magma code of Aurel Page to compute a fundamental domain for the action of $\Gamma$ on the upper half-space $\uhs$. Given this, $\h^1(\Gamma,\Z)$ is the abelianisation of $\Gamma$, and distinguished classes are constructed by computing the kernels of sufficiently many operators $T_{\mathfrak{q}} - \alpha_{\mathfrak{q}}.$ 

In practice, working with Bianchi groups is a difficult computational problem, and one that greatly restricts the scope of computations. This is particularly the case in the base-change setting. Consider, for example, the smallest elliptic curve of rank 1, which has conductor 37, and say we want to compute the $3$-adic $L$-function; then we must work with the group $\Gamma_0(3\cdot 37\roi_K)$, which has index $\sim 12,000$ in $\mathrm{SL}_2(\roi_K)$. The first rank 2 example has conductor $389$, leading to index $\sim 1,300,000$, far beyond current state-of-the-art (for example, in \cite{LMFDB}, the highest conductors appearing for Bianchi groups are of index 500,000).

To counter this computational difficulty, we used ...[DESCRIBE? OR LEAVE OUT?]



\subsection{Computing with distributions}
Two variable distributions with coefficients in an extension $L/\Qp$ are naturally in bijection with doubly-indexed bounded sequences in $L$ (see \cite[Proposition 3.6]{Wil17}), the map being given by
\[
 \mu \mapsto \{\mu(x_iy_j) : i,j \geq 0\}. 
\]
We can compute with these distributions using the \emph{finite approximation modules} \emph{op.\ cit.}.

%%%%%%%%%%%%%%%%%%%%%%%%%%%%%%%%%%%%%%%%%%%%%%%%%%%%%%%%%%%%%%%%%%%%
\subsubsection{Computing the action of \texorpdfstring{$\Sigma_0(p)^2$}{S0(p)2} on distributions}

It is important to explicitly understand the action of $\Sigma_0(p)^2$ on these distributions. In particular, there are an increasing number of conventions regarding actions (both left and right) on distributions, and we wish to be as clear as possible about the ones we adopt.
\begin{definition}
Consider the usual basis $\{x^iy^j : i,j \geq 0\}$ of $\A(L)$, and order it first by total degree $i+j$ and then lexicographically in $j$ and then $i$. To make this explicit, we label these basis monomials by defining
\[
	v_{n,i} \defeq x^{n-i}y^{i},
 \]
which are then ordered $v_{0,0}, v_{1,0},v_{1,1},v_{2,0},v_{2,1},\dots$, corresponding to $1,x,y,x^2,xy,\dots$ .
\end{definition}
Recall that $\Sigma_0(p)^2$ acts on the \emph{left} of $\A(L)$ by
\[
	g \cdot f(x,y) = f\left(\frac{b+dx}{a+cx},\frac{\overline{b}+\overline{d}x}{\overline{a} + \overline{c}x}\right), \hspace{12pt} g = \left[\matrd{a}{b}{c}{d},\matrd{\overline{a}}{\overline{b}}{\overline{c}}{\overline{d}}\right].
\]
Denote by $\psi_{\A}(g)$ the matrix of $g$ acting on $\A(L)$ in the basis $
\{v_{n,i}\}$.

We want to compute the dual action on distributions. The natural dual action is a \emph{right} action. However, since we will ultimately compute with the group cohomology -- which prefers \emph{left} actions -- it is convenient here to pass to the right action on $\A_2(L)$ defined by $f|g = g^{-1}\cdot f,$ inducing a left dual action. 

Now write
\[
	v_{n,i}^* \defeq \mathcal{X}^{n-i}\mathcal{Y}^i
\]
for the dual basis, where $v_{n,i}^*(v_{m,j}) = \delta_{mn}\delta_{ij}.$ For , then the matrix of $g$ acting on $\D(L)$, on the \emph{left}, in this dual basis, is given by 
\[
	\psi_{\D}(g) \defeq \psi_{\A}(g^{-1})^T.
\]
We are reduced, then, to computing the action on $\A$.

In practice, we use the following algorithm. Define 
\[
	r = \frac{b+dx}{a+cx}, \hspace{12pt} s = \frac{\overline{b}+\overline{d}y}{\overline{a}+\overline{c}x}.
\]
We then start by considering $r^0s^0 = 1$. Suppose we have computed $r^{n-i}s^i$ for all $i \in \{0,...,n\}.$ The coefficients of $r^{n-i}s^i$ give the column (at $v_{n,i}$) of the matrix of $g$ acting on $\A$. We then compute at step $n+1$ by multiplying by $r$ or $s$ as necessary, computing up to the desired precision.


\subsection{An explicit lifting theorem}
Including commutative diagram relating arithmetic cohomology and modular symbols with the action of the $U_{\pri}, U_{\pribar}$ operators. (Not well-defined on cocycles in general).


Complexity in computing this: lots of Up operators. Fox derivative and gradient.


\section{Data}
[Include our data, in some form or other]


\section*{To do:}

\begin{enumerate}\setlength{\itemsep}{-2pt}
    \item Sketch of sharp/flat construction
    \item description of how to get the classical class/citation of Marc's papers
    \item More detail on the pseudo-nullity conjecture
    \item description of resultant calculations to show coprimality
    \item Evidence!!!
    \item Fox derivatives and Fox gradients (RH Fox, free differential calculus)
    \item Tietze algorithm
\end{enumerate}

\footnotesize
\renewcommand{\refname}{\normalsize References} 
\bibliography{references}{}
\bibliographystyle{alpha}

\Addresses

\end{document}

